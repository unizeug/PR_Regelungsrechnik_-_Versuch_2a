\newcommand{\institut}{}
\newcommand{\fachgebiet}{Regelungstechnik}
\newcommand{\veranstaltung}{Praktikum Grundlagen der Regelungstechnik}
\newcommand{\pdfautor}{Dirk Barbendererde (321 836), Boris Henckell (325 779)}
\newcommand{\autor}{Dirk Barbendererde (321 836)\\ Boris Henckell (325 779)}
\newcommand{\pdftitle}{Praktikum\ Regelungstechnik\ Versuch\ 2a}
\newcommand{\prototitle}{Praktikum Regelungstechnik \\ Versuch 2a}
\newcommand{\aufgabe}{}

\newcommand{\gruppe}{Gruppe: G1 Di 12-14}
\newcommand{\betreuer}{Betreuer: Behrang Monajemi Nejad}



\input{../../packages/tu_header_8}
%\begin{document}

% \lstlistoflistings
\definecolor{darkgray}{rgb}{0.95,0.95,0.95}
\lstset{language=Scilab}
\lstset{inputencoding=utf8}
%\lstset{extendedchars=true} % Umlaute an der richtigen stelle und nicht am Anfang ausgeben
\lstset{backgroundcolor=\color{darkgray}}
\lstset{numbers=left, numberstyle=\tiny, stepnumber=1, numbersep=7pt, breaklines=true}
\lstset{keywordstyle=\color{red}\bfseries\emph}
\lstset{
breaklines,
numbers=left,
frame=single,
xleftmargin=-2cm,
xrightmargin=-1.5cm
}
% enables UTF-8 in source code: (dirty, dirty hack)
\lstset{literate=
    %Deutsch
    {ä}{{\"a}}1 {ö}{{\"o}}1 {ü}{{\"u}}1 {Ä}{{\"A}}1 {Ö}
    {{\"O}}1 {Ü}{{\"U}}1 {ß}{\ss}1
    %Türkisch
    {â}{{\^{a}}}1 {Â}{{\^{A}}}1 {ç}{{\c{c}}}1 {Ç}{{\c{C}}}1 {ğ}{{\u{g}}}1 {Ğ}{{\u{G}}}1 {ı}{{\i}}1 {İ}{{\.{I}}}1 {ö}{{\"o}}1 {Ö}{{\"O}}1 {ş}{{\c{s}}}1
    {Ş}{{\c{S}}}1 {ü}{{\"u}}1 {Ü}{{\"U}}1
    %Polish
    {ą}{{\k{a}}}1 {ć}{{\'c}}1 {ę}{{\k{e}}}1 {ł}{{\l{}}}1 {ń}{{\'n}}1 {ó}{{\'o}}1 {ś}{{\'s}}1 {ż}{{\.z}}1 {ź}{{\'z}}1 {Ą}{{\k{A}}}1 {Ć}{{\'C}}1
    {Ę}{{\k{E}}}1 {Ł}{{\L{}}}1 {Ń}{{\'N}}1 {Ó}{{\'O}}1 {Ś}{{\'S}}1 {Ż}{{\.Z}}1 {Ź}{{\'Z}}1
    %Spanish
    {á}{{\'a}}1 {é}{{\'e}}1 {í}{{\'i}}1 {ó}{{\'o}}1 {ú}{{\'u}}1 {ñ}{{\~n}}1
}

%     \lstinputlisting{./praktikum6.sce}



%---------------------------------------------------------------------
%---------------------------------------------------------------------
%---------------------------------------------------------------------

\section{Vorbereitungsaufgaben}
\begin{quote}
	\hspace{-2em}
	\subsection{Linearisierung}
    Aufgabe:\\
    Linearisieren Sie das nichtlineare Modell um die Ruhelage ($z, \varphi$) = ($0, 0$) und berechnen Sie die
    Transferfunktion von der Stellgröße $u$ zum Ausgang $\varphi$. Skizzieren Sie die Pol-/Nullstellenverteilung des
    Systems.
	\begin{quote}
	   Das folgende Systemmodell war in dem Aufgabenblatt gegeben:
	   \begin{equation*}
        	\begin{split}
        		\dot{z} &= u\\
        		\ddot{z} &=\frac{\mathrm d\dot{z}}{\mathrm d t} = \frac{\mathrm du}{\mathrm d t}\\
        		\ddot{\varphi} &= \frac{1}{J_s + m_P a^2} ( m_P a (\ddot{z}cos(\varphi) +g sin(\varphi))-c\dot{\varphi})
        	\end{split}
        \end{equation*}
        
        Als Ziel der Linearisierung streben wir eine Übertragungsfunktion der Form 
        \begin{equation*}
        	\begin{split}
        		\frac{Y(s)}{U(s)}
        	\end{split}
        \end{equation*}
        
        an. Da unser Regelgröße $Y(s)$ der Winkel $\varphi(s)$ ist und unsere Stellgröße $U(s)$ die Wagengeschwindigkeit
        $\dot{Z}(s)$ versuchen wir das bekannte Systemmodell nur in abhängigkeit dieser beiden Größen darzustellen.\\
        Dazu definieren wir:
        \begin{equation*}
        	\begin{split}
        		\ddot{\varphi} = f(\ddot{z}, \varphi, \dot{\varphi})
        	\end{split}
        \end{equation*}
        
        Nun beginnen wir mit der Linearisierung.
        
        \begin{equation*}
        	\begin{split}
        		\Delta \ddot{\varphi} &= \left \frac{\delta f}{\delta \ddot{z}} \Delta \ddot{z} \right|_{AP} + \left
        		\frac{\delta f}{\delta \varphi} \Delta \varphi \right|_{AP} + \left \frac{\delta f}{\delta \dot{\varphi}}
        		\Delta \dot{z} \right|{AP}\\
        		&= \frac{m_p a \cdot cos(0)}{J_s + m_P a^2} \cdot \Delta \ddot{z} + \frac{m_P a}{J_s + m_P a^2} (-\ddot{z}
        		\cdot sin(0) + g \cdot cos(0)) \cdot \Delta \varphi - \frac{c}{J_s + m_P a^2} \cdot \Delta \dot{\varphi}\\
        		&= \frac{1}{J_s + m_P a^2} \left ( m_p a \Delta \cdot \ddot{z} + m_P a g \Delta \cdot \varphi - c \Delta
        		\cdot \dot{\varphi} \right)\\
        	\end{split}
        \end{equation*}
        
        Diese Gleichung transformieren wir mittels Laplace.
        
        \begin{equation*}
        	\begin{split}
        		s^2 \Delta \varphi - s \varphi(0) - \dot{\varphi}(0) &= \frac{1}{J_s + m_P a^2} \left (  m_p a (s \Delta
        		\dot{Z} - Z(0)) + m_p a g \Delta \varphi - c (s \Delta \varphi - \varphi(0)) \right)\\
        		\\
        		\Delta \varphi (s^2 + c s - \frac{m_P a g}{J_s + m_P a^2}) &= \frac{m_p a s}{J_s + m_P a^2} \Delta \dot{Z}\\
        		\\
        		\frac{\Delta \varphi}{\Delta \dot{Z}} = \frac{Y(s)}{U(s)} &= \frac{\frac{m_p a s}{J_s + m_P a^2}}{s^2 + c s -
        		\frac{m_P a g}{J_s + m_P a^2}}\\
        		\\
        		\frac{Y(s)}{U(s)} &= \frac{m_p a s}{(J_s + m_p a^2) s^2 + s c - m_p a g}
        	\end{split}
        \end{equation*}
        
        \TODO{Dirk: Bitte die Skizze der Pol-Nullstellenverteilung der Strecke einfügen} \\
        
	\end{quote}
	
	\subsection{Wurzelortskurve}
    Aufgabe:\\
    Machen Sie sich mit den Konstruktionsregeln für Wurzelortskurven vertraut! Gegeben sei der folgende Regler\\
    \begin{equation*}
        \begin{split}
            G_r(s) = K\frac{s + \alpha}{s + \beta}, \ \ K,\alpha, \beta \in  \Re
        \end{split}
    \end{equation*}
    mit $K > 0$. Entscheiden Sie anhand des Wurzelortskurvenverfahrens über die notwendige Lage der Reglerpolstelle
    und -nullstelle, damit das geregelte System für eine genügend große Verstärkung $K$ stabilisiert werden kann.
    Bestimmen Sie nur, in welcher Hälfte der s-Ebene jeweils die Pol- und die Nullstelle liegen muss und nicht die
    genaue Position.
	\begin{quote}
	
	\end{quote} %Ende Subsection WOK
	
	\subsection{Anti-Windup-Schaltung in Scicos erstellen}
	Aufgabe:\\
    Erweitern Sie Ihre Reglerstruktur in Scicos um eine Anti-Windup-Schaltung. Beziehen Sie sowohl
    den äußeren als auch den inneren Regler in die Schaltung ein. Realisieren Sie dafür den inneren und äußeren
    PI-Regler in einer Parallelform mit P- und I-Anteil.
	\begin{quote}
		
		
	\end{quote}
	
\end{quote}

%--------------------------------------------------------------------
%--------------------------------------------------------------------

\section{Ergebnisse}
\begin{quote}
    
    
    \subsection{Regler mit guten Störeigenschaften}
    Implementieren Sie ihre Kaskadenregelung mit innerem und äußerem Regler und der
    Anti-Windup-Schaltung in Scicos und erstellen Sie mit dem Betreuer das echtzeitfähige
    Programm zur Motoransteuerung. Verwenden Sie für den äußeren Regler zunächst die
    Parameter aus Aufgabe \ref{2f}.
    
    
    \begin{quote}
        
        \subsubsection{Führungssprungantwort}
        Nehmen Sie die Führungssprungantwort des Drehzahlregelkreises auf. Schalten Sie dafür den Sollwert der
        Winkelgeschwindigkeit von $0 \mathrm{\frac{rad}{s}}$ auf $150 \mathrm{\frac{rad}{s}}$ Nehmen Sie zusätzlich zu der
        Regelgröße auch die Stellgröße $r_i$ des Regelkreises auf.
        Beschreiben und diskutieren Sie das Regelkreisverhalten! Vergleichen Sie das Ergebnis mit der Simulation des
        Führungsverhaltens aus der Vorbereitungsaufgabe \ref{2f}!
        
        \begin{quote}
            
        \begin{center}
        \begin{tabular}{ll}
        
        \hspace{-4.5cm}
            \begin{minipage}{0.6\textwidth}
                
                \begin{figure}[H]
                    \label{fig:sprung_w}
                    \includegraphics[scale=0.7, trim = 8mm 5mm 15mm 10mm, clip]{Bilder/sprung_w}
                    \caption{Sprungantwort der Winkelgeschwindigkeit}
                \end{figure}
                
            \end{minipage}
            
            \begin{minipage}{0.6\textwidth}
                \begin{figure}[H]
                    \label{fig:sprung_i}
                    \includegraphics[scale=0.7, trim = 8mm 5mm 15mm 10mm, clip]{Bilder/sprung_i}
                    \caption{Sprungantwort des Stroms}
                \end{figure}
                
            \end{minipage}
            
        \end{tabular}
        \end{center}
        
        \vspace{2em}
        
        Die Sprungantworten erfüllen alle gestellten Bedingungen. Die Geschwindigkeit ändert sich etwas langsamer als
        simuliert, erreicht jedoch schon nach \si{0,35}{s} den angestrebten Wert von $150 \mathrm{\frac{rad}{s}}$. Die geringe
        Abweichung ist mit höherer Trägheit und mehr Reibmoment gegenüber der Simulation zu erklären.\\
        Auch die Sprungantwort des Stroms stellt sich mit nur geringem Überschwingen sehr schnell wieder ein.
        
        
        
        \end{quote}
        
        
        \subsubsection{Störsprungantwort}
        Nehmen Sie die Störsprungantwort des Systems auf. Betreiben Sie den Motor bei einer Winkelgeschwindigkeit von $150
        \mathrm{\frac{rad}{s}}$ und lösen Sie die Rückhaltevorrichtung der Bremse um ein sprunghaftes Lastmoment zu erhalten.
        Nehmen Sie zusätzlich zu der Regelgröße auch die Stellgröße $r_i$ des Regelkreises auf. Beschreiben und diskutieren Sie
        das Regelkreisverhalten! Vergleichen Sie das Ergebnis mit der Simulation des Störverhaltens aus der Vorbereitungsaufgabe
        \ref{2f}!
        
        \begin{quote}
            
        \begin{center}
        \begin{tabular}{ll}
        
        \hspace{-4.5cm}
            \begin{minipage}{0.6\textwidth}
                
                \begin{figure}[H]
                    \label{fig:stoersprung_w}
                    \includegraphics[scale=0.7, trim = 8mm 5mm 15mm 10mm, clip]{Bilder/stoersprung_w}
                    \caption{Störsprungantwort der Winkelgeschwindigkeit}
                \end{figure}
                
            \end{minipage}
            
            \begin{minipage}{0.6\textwidth}
                \begin{figure}[H]
                    \label{fig:stoersprung_i}
                    \includegraphics[scale=0.7, trim = 8mm 5mm 15mm 10mm, clip]{Bilder/stoersprung_i}
                    \caption{Störsprungantwort des Stroms}
                \end{figure}
                
            \end{minipage}
            
        \end{tabular}
        \end{center}
        \vspace{2em}
        
        Wie auch schon bei letzten Versuch beobachtet reagiert die Störsprungantwort des Reglers sogar etwas schneller als wir es
        simuliert haben. Dies hängt wiederum mit ungenauigkeiten der Realen bauteile zusammen, welche sich in diesem Fall zu
        unseren Gunsten auswirken.
        
        
        \end{quote}
    
    \end{quote}
    
    
    \subsection{Regler mit schlechten Störeigenschaften}
    Untersuchen Sie nun den Regelkreis mit den zuerst bestimmten Parametern aus Aufgabe
    \ref{2d}. Nehmen Sie die Störsprungantwort des Systems auf. Wie äußert sich das unbefriedigende Störverhalten in der Praxis?
    Vergleichen Sie das Ergebnis mit der Simulation des Störverhaltens aus der Vorbereitungsaufgabe \ref{2e}!
    
    \begin{quote}
        
        Dieser Versuchsteil hat leider nicht den gewünschten bzw. erwarteten Erfolg gebracht. Erwartet hätten wir einen
        ähnlich schnelle Führungssprungantwort wie mit dem guten Regler jedoch eine sehr viel langsamere
        Störsprungantwort. Der Versuch hat jedoch schon eine sehr langsame Führungssprungantwort, mit
        einer Ausregelzeit weit größer als $20 s$, ergeben. Laut unserer Simulation hätte sich hingegen eine Ausregelzeit von
        unter $0.1 s$ einstellen sollen. Den Fehler für diese Abweichung konnten wir nicht identifizieren.\vspace{1em}
        
        Unerklärlich war außerdem, dass die Änderung des Vorzeichens in der Rückführung des PT1-Gliedes im Scicos Modell
        keine Auswirkung auf das Regelverhalten hatte.\vspace{1em}
        
        Da wir leider keine vernünftigen Messergebnisse von diesem Versuch haben können wir weder etwas plotten noch
        analysieren.
        
        
    \end{quote}
    
    
    \subsection{Test der Anti-Windup-Schaltung}
    Testen Sie ihre Anti-Windup-Schaltung mit den Parametern aus Aufgabe \ref{2f}. Betreiben Sie den Motor bei einer
    Winkelgeschwindigkeit von $150 \mathrm{\frac{rad}{s}}$ und halten Sie die Scheibe sehr kurz fest um sie danach sofort wieder
    freizugeben. Wiederholen Sie den Versuch mit deaktivierter Anti-Windup-Schaltung! Nehmen Sie neben der Winkelgeschwindigkeit
    und dem Ankerstrom die Stellgrößen vor und nach der Beschränkung auf. Beschreiben und vergleichen Sie das Regelkreisverhalten
    mit und ohne Anti-Windup-Schaltung!
    
    \begin{quote}
        
        Bei diesem Versuch haben wir den laufenden Motor nicht nur durch eine Last in seiner Umdrehung gestört sondern ihn durch
        festhalten der Schwungscheibe kurz angehalten. Dadurch haben wir den Regler gezwungen so stark auf diese Störung zu
        reagieren, dass er mit seiner inneren Stellgröße $u$ in die Begrenzung von \si{\pm5}{V} geht. Diesen Versuch haben wir
        sowohl mit eingeschalteter als auch mit ausgeschalteter Wind-up Schaltung durchgeführt.\\
        Die Messergebniss der zwei Versuche sehen Folgendermaßen aus:

        \begin{figure}[H]
        \centering
            \label{fig:w}
            \includegraphics[scale=0.8, trim = 0.5cm 0.5cm 2cm 0.5cm, clip]
            {./Bilder/windup_sprungantwort_Geschwindigkeit}
            \caption{Störsprungantwort der Geschwindigkeit}
        \end{figure}
        
        \begin{figure}[H]
        \centering
            \includegraphics[scale=0.8, trim = 0.5cm 0.5cm 2cm 0.5cm, clip]
            {./Bilder/windup_sprungantwort_sollstrom}
            \caption{Störsprungantwort des Sollstroms}
        \end{figure}
        
        \begin{figure}[H]
        \centering
            \includegraphics[scale=0.8, trim = 0.5cm 0.5cm 2cm 0.5cm, clip]
            {./Bilder/windup_sprungantwort_Spannung}
            \caption{Störsprungantwort der Spannung}
        \end{figure}
        

        \begin{figure}[H]
        \centering
            \includegraphics[scale=0.8, trim = 0.5cm 0.5cm 2cm 0.5cm, clip]
            {./Bilder/windup_sprungantwort_Strom}
            \caption{Störsprungantwort des Stroms}
        \end{figure}
        
        In der Grafik \ref{fig:w} sieht man klar, wann die Schwungscheibe 
        von $150 \frac{rad}{s}$ auf $0\frac{rad}{s}$ abgebremst wird. Als reaktion darauf versucht
        der Regler den Motor anzuregen. Dazu steigt der Sollstrom an, weshalb wiederum die Spannung sprungartig
        ansteigt. Hierbei wiederum gerät die Spannung in die Begrenzung.\\
        
        \subsubsection{aktive anti-Windup Schaltung}
		\begin{quote}
            Im Falle der aktivierten anti-Wind-up Schaltung bremst diese sofort die Intregratoren der beiden
            Stellgrößen, sodass nicht über die physikalische Beschränkung hinaus integriert wird. Dadurch stimmen die
            physikalischen Größen und die Stellgrößen weiterhin überein. Dies verhindert, dass die Regler mit ihren Reglgrößen
            über ihr Ziel hinausschießen.\\
            Dieser Effekt lässt sich sehr gut an den Grünen Linien in der Sprungantwort der Spannung sowie der
            Sprunganwort des Sollstroms erkennen.\\
            Auch an der Drehgeschwindigkeit lässt sich dieses abbremsen erkennen, da die Grüne Linie langsamer aus der
            Störung rauskommt als die Blaue. Dafür schafft sie es jedoch, sobald die gewünschte Drehzahl erreicht ist, die
            Beschleunigung komplett abzubremsen und hat somit die Störung ausgeregelt.\vspace{1em}
			
		\end{quote}
		
		\subsubsection{deaktive anti-Winfup Schaltung}
		\begin{quote}
            Im Gegensatz dazu versucht der Regler ohne anti-Windup die Störung auszugleichen als ob es keine Begrenzung gäbe. Das
            wiederum hat zur Folge, dass der Regler annimmt der Ankerstrom würde nicht ausreichen um gegenzusteuern und
            deshalb die Stellgröße weiter aufintegriert.\\
            Sobald der Regler sich der angestrebten Winkelgeschwindigkeit nähert, müsste er anfangen gegenzusteuern um 
            Überschwingen zu vermeiden. Da der Integrator des inneren Reglers inzwischen aber einen sehr hohen Wert angenommen
            hat, hat das gegensteuern zwar einen Einfluss auf den Integrator bzw. die theoretische Stellgröße, nicht jedoch auf
            die physikalische.\\
            Durch das fehlende Gegensteuern schwingt die Regelgröße über. Der Regler benötigt somit, falls er es überhaup schafft,
            sehr lange um die ursprüngliche Störung auszuregeln.\\
            Das Überschwingen lässt sich in allen vier Messerten erkennen.
		\end{quote}
        
    \end{quote}


    
    
    
    
\end{quote}

%--------------------------------------------------------------------
%--------------------------------------------------------------------



% \begin{quote}
%     \lstinputlisting[
%         caption={Scilab-script},
%         label=lst:scilab]
%         {./Scilab/Motor.sce}
%         
% \end{quote}

%--------------------------------------------------------------------
%--------------------------------------------------------------------
\begin{thebibliography}{999}
\bibitem {Ueberschwingweite} Prof. Dr.-Ing. Raisch, Jörg; Dipl.-Ing. Hess, Anne-Katrin; Dipl.-Ing. Seel, Thomas:
Grundlagen der Regelungstechnik - 4.Praktikum, S.5
\bibitem {Ausregelzeit} Prof. Dr.-Ing. Raisch, Jörg; Dipl.-Ing. Hess, Anne-Katrin; Dipl.-Ing. Seel, Thomas:
Grundlagen der Regelungstechnik - 4.Praktikum, S.5

%\usepackage{url}


\bibitem{krachler}Christian Krachler:
\href{http://www.krachler.com/fileadmin/user_upload/arbeiten/Reglersynthese_Christian_Krachler.pdf}{Reglersynthese nach dem Frequenzkennlinienverfahren}, S16, S22, 08.05.2012

%http://krachler.com/fileadmin/user\_upload/arbeiten/Reglersynthese\_Christian\_Krachler.pdf


%Name, Vorname.; evtl. Name2, Vorname2.: Titel des Dokumentes
%oder Buches, Zeitschrift/Verlag/URL (Auflage, Erscheinungsort, -jahr), ggf. Seitenzahlen
%\bibitem [Wiki10] {DigitaleMesskette2} \url{www.wikipedia.org}, Zugriff 22.03.2010
\end{thebibliography}


\end{document}
