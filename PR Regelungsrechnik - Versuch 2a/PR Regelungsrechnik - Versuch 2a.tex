\newcommand{\institut}{}
\newcommand{\fachgebiet}{Regelungstechnik}
\newcommand{\veranstaltung}{Praktikum Grundlagen der Regelungstechnik}
\newcommand{\pdfautor}{Dirk Barbendererde (321 836), Boris Henckell (325 779)}
\newcommand{\autor}{Dirk Barbendererde (321 836)\\ Boris Henckell (325 779)}
\newcommand{\pdftitle}{Praktikum\ Regelungstechnik\ Versuch\ 2a}
\newcommand{\prototitle}{Praktikum Regelungstechnik \\ Versuch 2a}
\newcommand{\aufgabe}{}

\newcommand{\gruppe}{Gruppe: G1 Di 12-14}
\newcommand{\betreuer}{Betreuer: Behrang Monajemi Nejad}



\input{../../packages/tu_header_8}
%\begin{document}

% \lstlistoflistings
\definecolor{darkgray}{rgb}{0.95,0.95,0.95}
\lstset{language=Scilab}
\lstset{inputencoding=utf8}
%\lstset{extendedchars=true} % Umlaute an der richtigen stelle und nicht am Anfang ausgeben
\lstset{backgroundcolor=\color{darkgray}}
\lstset{numbers=left, numberstyle=\tiny, stepnumber=1, numbersep=7pt, breaklines=true}
\lstset{keywordstyle=\color{red}\bfseries\emph}
\lstset{
breaklines,
numbers=left,
frame=single,
xleftmargin=-2cm,
xrightmargin=-1.5cm
}
% enables UTF-8 in source code: (dirty, dirty hack)
\lstset{literate=
    %Deutsch
    {ä}{{\"a}}1 {ö}{{\"o}}1 {ü}{{\"u}}1 {Ä}{{\"A}}1 {Ö}
    {{\"O}}1 {Ü}{{\"U}}1 {ß}{\ss}1
    %Türkisch
    {â}{{\^{a}}}1 {Â}{{\^{A}}}1 {ç}{{\c{c}}}1 {Ç}{{\c{C}}}1 {ğ}{{\u{g}}}1 {Ğ}{{\u{G}}}1 {ı}{{\i}}1 {İ}{{\.{I}}}1 {ö}{{\"o}}1 {Ö}{{\"O}}1 {ş}{{\c{s}}}1
    {Ş}{{\c{S}}}1 {ü}{{\"u}}1 {Ü}{{\"U}}1
    %Polish
    {ą}{{\k{a}}}1 {ć}{{\'c}}1 {ę}{{\k{e}}}1 {ł}{{\l{}}}1 {ń}{{\'n}}1 {ó}{{\'o}}1 {ś}{{\'s}}1 {ż}{{\.z}}1 {ź}{{\'z}}1 {Ą}{{\k{A}}}1 {Ć}{{\'C}}1
    {Ę}{{\k{E}}}1 {Ł}{{\L{}}}1 {Ń}{{\'N}}1 {Ó}{{\'O}}1 {Ś}{{\'S}}1 {Ż}{{\.Z}}1 {Ź}{{\'Z}}1
    %Spanish
    {á}{{\'a}}1 {é}{{\'e}}1 {í}{{\'i}}1 {ó}{{\'o}}1 {ú}{{\'u}}1 {ñ}{{\~n}}1
}

%     \lstinputlisting{./praktikum6.sce}



%---------------------------------------------------------------------
%---------------------------------------------------------------------
%---------------------------------------------------------------------

\section{Vorbereitungsaufgaben}
\begin{quote}
	\hspace{-2em}
	\subsection{Linearisierung}
    Aufgabe:\\
    Linearisieren Sie das nichtlineare Modell um die Ruhelage ($z, \varphi$) = ($0, 0$) und berechnen Sie die
    Transferfunktion von der Stellgröße $u$ zum Ausgang $\varphi$. Skizzieren Sie die Pol-/Nullstellenverteilung des
    Systems.
	\begin{quote}
	   Das folgende Systemmodell war in dem Aufgabenblatt gegeben:
	   \begin{equation*}
        	\begin{split}
        		\dot{z} &= u\\
        		\ddot{z} &=\frac{\mathrm d\dot{z}}{\mathrm d t} = \frac{\mathrm du}{\mathrm d t}\\
        		\ddot{\varphi} &= \frac{1}{J_s + m_P a^2} ( m_P a (\ddot{z}cos(\varphi) +g sin(\varphi))-c\dot{\varphi})
        	\end{split}
        \end{equation*}
        
        Als Ziel der Linearisierung streben wir eine Übertragungsfunktion der Form 
        \begin{equation*}
        	\begin{split}
        		\frac{Y(s)}{U(s)}
        	\end{split}
        \end{equation*}
        
        an. Da unser Regelgröße $Y(s)$ der Winkel $\varphi(s)$ ist und unsere Stellgröße $U(s)$ die Wagengeschwindigkeit
        $\dot{Z}(s)$ versuchen wir das bekannte Systemmodell nur in abhängigkeit dieser beiden Größen darzustellen.\\
        Dazu definieren wir:
        \begin{equation*}
        	\begin{split}
        		\ddot{\varphi} = f(\ddot{z}, \varphi, \dot{\varphi})
        	\end{split}
        \end{equation*}
        
        Nun beginnen wir mit der Linearisierung.
        
        \begin{equation*}
        	\begin{split}
        		\Delta \ddot{\varphi} &= \left \frac{\delta f}{\delta \ddot{z}} \Delta \ddot{z} \right|_{AP} + \left
        		\frac{\delta f}{\delta \varphi} \Delta \varphi \right|_{AP} + \left \frac{\delta f}{\delta \dot{\varphi}}
        		\Delta \dot{z} \right|{AP}\\
        		&= \frac{m_p a \cdot cos(0)}{J_s + m_P a^2} \cdot \Delta \ddot{z} + \frac{m_P a}{J_s + m_P a^2} (-\ddot{z}
        		\cdot sin(0) + g \cdot cos(0)) \cdot \Delta \varphi - \frac{c}{J_s + m_P a^2} \cdot \Delta \dot{\varphi}\\
        		&= \frac{1}{J_s + m_P a^2} \left ( m_p a \Delta \cdot \ddot{z} + m_P a g \Delta \cdot \varphi - c \Delta
        		\cdot \dot{\varphi} \right)\\
        	\end{split}
        \end{equation*}
        
        Diese Gleichung transformieren wir mittels Laplace.
        
        \begin{equation*}
        	\begin{split}
        		s^2 \Delta \varphi - s \varphi(0) - \dot{\varphi}(0) &= \frac{1}{J_s + m_P a^2} \left (  m_p a (s \Delta
        		\dot{Z} - Z(0)) + m_p a g \Delta \varphi - c (s \Delta \varphi - \varphi(0)) \right)\\
        		\\
        		\Delta \varphi (s^2 + c s - \frac{m_P a g}{J_s + m_P a^2}) &= \frac{m_p a s}{J_s + m_P a^2} \Delta \dot{Z}\\
        		\\
        		\frac{\Delta \varphi}{\Delta \dot{Z}} = \frac{Y(s)}{U(s)} &= \frac{\frac{m_p a s}{J_s + m_P a^2}}{s^2 + c s -
        		\frac{m_P a g}{J_s + m_P a^2}}\\
        		\\
        		\frac{Y(s)}{U(s)} &= \frac{m_p a s}{(J_s + m_p a^2) s^2 + s c - m_p a g}
        	\end{split}
        \end{equation*}
        
        \TODO{Dirk: Bitte die Skizze der Pol-Nullstellenverteilung der Strecke einfügen} \\
        
	\end{quote} %Ende Subsection Linearisierung
	
	\subsection{Wurzelortskurve}
    Aufgabe:\\
    Machen Sie sich mit den Konstruktionsregeln für Wurzelortskurven vertraut! Gegeben sei der folgende Regler\\
    \begin{equation*}
        \begin{split}
            G_r(s) = K\frac{s + \alpha}{s + \beta}, \ \ K,\alpha, \beta \in  \Re
        \end{split}
    \end{equation*}
    mit $K > 0$. Entscheiden Sie anhand des Wurzelortskurvenverfahrens über die notwendige Lage der Reglerpolstelle
    und -nullstelle, damit das geregelte System für eine genügend große Verstärkung $K$ stabilisiert werden kann.
    Bestimmen Sie nur, in welcher Hälfte der s-Ebene jeweils die Pol- und die Nullstelle liegen muss und nicht die
    genaue Position.
	\begin{quote}
	
	\end{quote} %Ende Subsection WOK
	
	\subsection{Nyquistkriterium}
	Aufgabe:\\
    Machen Sie sich mit dem allgemeinen Nyquistkriterium vertraut! Skizzieren Sie mit der zuvor bestimmten
    Pol-/Nullstellenverteilung des Reglers das Nyquistdiagramm des offenen Regelkreises. Gegebenenfalls kann eine
    konkrete Realisierung des Reglers angenommen werden, um das Nyquistdiagramm zeichnen zu können. Welche Bedingungen
    werden nach dem Nyquistkriterium an die Ortskurve der offenen Kette hinsichtlich der Stabilität gestellt und wie
    kann man dieses Bedingungen im Bodediagramm wiederfinden?
	\begin{quote} 
	   \TODO{Dirk: bitte eine Skizze von einer passenden Nyquistortskurve reinstellen} \\
        Anhand der Nyquistortskurve lässt sich die Stabilität eines geschlossenen Regelkreises für eine feste
        Verstärkung ermitteln. Dazu darf die Nyquistortskurve nicht durch den kritischen Punkt ($-1,0$) laufen und muss
        die Phasendrehung bezüglich des kritischen Punktes genau $\pi \cdot (r_G + r_K)$ betragen.\\
        $r_G$ und $r_K$ sind in diesem fall die Anzahl der Polstellen mit positiven Realteil der Strecke bzw des
        Reglers.
        Außerdem lassen wir die Nyquistorstkurve von $0$ bis $\infty$ laufen. Würden wir sie von $-\infty$ bis $\infty$
        laufen lassen müsste die Phasendrehung $2\pi (r_G + r_K)$ sein.\vspace{1em}
		
		In unserem Fall haben wir eine Polstelle der Strecke und eine Polstelle des Reglers mit positivem Realteil. Daher
		benötigen wir eine Nyquistortskurve die nicht durch den kritischen Punkt läuft und eine Phasendrehung von $2\pi$ hat um ein stabilen geschlossenen Regelkreis zu haben.\\
		Diese Bedingung prüfen wir bei dem Reglerentwurf im nächsten Aufgabenabschnitt.\vspace{1em}
		
		Nachdem die Stabitität des geschlossenen Regelkreises bestätigt wurde lässt sich an der Nyquistortskurve noch weitere
		Informationen, wie die Phasenreserve und die Amplitudenreserve, ablesen. Die Phasenreserve des Regelkreises erkennt
		man an dem Winkel zwischen dem Schnittpunkt der Nyquistortskurve und dem Einheitskreis und der negativen reellen
		Achse. Die Amplitudenreserve widerum erkennt man an dem Kehrwert des Abstandes zwischen der y-Achse und dem
		Schnittpunkt der Ortskurve mit der negativen reellen Achse.\\
		Die Phasenreserve und die Amplitudenreserve lassen sich auch an dem Bodediagramm ablesen. Diesesmal ist die
		Phasenreserve die Differenz der Phase zu $180^{\circ}$ bei der bzw. den Durchstrittsfrequenzen. Die Amplitudenreserve
		ist die Differenz des Amplitudenganges bezüglich $0dB$ bei der Frequenz von $180^{\circ}$.\vspace{1em}
		
		Wie oben beschrieben erwarten wir für unseren stabilen Regelkreis eine Phasendrehung von $2\pi$. Daraus resultiert
		automatisch, dass die Nyquistortskurve zweimal den Einheitskreis schneiden wird. Diese beiden Schnittpunkte mit dem
		Einheitskreis werden in dem Bodediagramm durch einen zweifache Schnittstelle mit der $0dB$ Achse im Amplitudengang zu
		sehen sein.
		
		
	\end{quote} %Ende Subsection Nyquist
	
	\subsection{Frequenzkennlinienentwurf}
    \begin{quote}
        
    \end{quote}  % Ende Subsection Frequenzkennlinienentwurf
    
    \subsection{Störverhalten}
    Aufgabe\\
    Untersuchen Sie das Verhalten des Regelkreises bei konstanten Ausgangsstörungen (z.B. durch Messfehler oder Wagen
    auf geneigter Ebene). Wie verhalten sich Wagenposition, Winkel und Wagengeschwindigkeit?
    \begin{quote}
        
    \end{quote}  % Ende Subsection Störverhalten
    
    \subsection{Führungsverhalten}
    Aufgabe\\
    Untersuchen Sie das Verhalten des Regelkreises gegenüber konstanten Sollvorgaben ungleich Null. Wie verhalten sich
    Wagenposition, Winkel und Wagengeschwindigkeit?
    \begin{quote}
        
    \end{quote}  % Ende Subsection Führungsverhalten
	
\end{quote} %Ende Section Vorbereitungsaufgaben 

%--------------------------------------------------------------------
%--------------------------------------------------------------------

\section{Ergebnisse}
\begin{quote}
 
\end{quote} %Ende Section Ergebnisse

%--------------------------------------------------------------------
%--------------------------------------------------------------------



% \begin{quote}
%     \lstinputlisting[
%         caption={Scilab-script},
%         label=lst:scilab]
%         {./Scilab/Motor.sce}
%         
% \end{quote}

%--------------------------------------------------------------------
%--------------------------------------------------------------------
\begin{thebibliography}{999}
%\bibitem {Ueberschwingweite} Prof. Dr.-Ing. Raisch, Jörg; Dipl.-Ing. Hess, Anne-Katrin; Dipl.-Ing. Seel, Thomas:
%Grundlagen der Regelungstechnik - 4.Praktikum, S.5
%\bibitem {Ausregelzeit} Prof. Dr.-Ing. Raisch, Jörg; Dipl.-Ing. Hess, Anne-Katrin; Dipl.-Ing. Seel, Thomas:
%Grundlagen der Regelungstechnik - 4.Praktikum, S.5

%\usepackage{url}


%\bibitem{krachler}Christian Krachler:
%\href{http://www.krachler.com/fileadmin/user_upload/arbeiten/Reglersynthese_Christian_Krachler.pdf}{Reglersynthese nach
% dem Frequenzkennlinienverfahren}, S16, S22, 08.05.2012

%http://krachler.com/fileadmin/user\_upload/arbeiten/Reglersynthese\_Christian\_Krachler.pdf


%Name, Vorname.; evtl. Name2, Vorname2.: Titel des Dokumentes
%oder Buches, Zeitschrift/Verlag/URL (Auflage, Erscheinungsort, -jahr), ggf. Seitenzahlen
%\bibitem [Wiki10] {DigitaleMesskette2} \url{www.wikipedia.org}, Zugriff 22.03.2010
\end{thebibliography}


\end{document}
